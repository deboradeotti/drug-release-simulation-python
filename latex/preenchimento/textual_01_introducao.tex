%%%%%%%%%%%%%%%%%%%%%%%%%%%%%%%%%%%%%%%%%%%%%%%%%%%%%%%%%%%%%%%%%%
% Aqui começa o capítulo de Introdução.
%%%%%%%%%%%%%%%%%%%%%%%%%%%%%%%%%%%%%%%%%%%%%%%%%%%%%%%%%%%%%%%%%%
% Use o comando \label para definir um rótulo, caso seja 
% necessário referenciar esse capítulo posteriormente.
%%%%%%%%%%%%%%%%%%%%%%%%%%%%%%%%%%%%%%%%%%%%%%%%%%%%%%%%%%%%%%%%%%
\chapter{Introdução}\label{chp:Introducao}
%%%%%%%%%%%%%%%%%%%%%%%%%%%%%%%%%%%%%%%%%%%%%%%%%%%%%%%%%%%%%%%%%%

\section{Contextualização}\label{sec:context}

Até o século XVIII, o pensamento médico da Europa acreditava na existência de um só sexo, o masculino. A mulher era o seu representante inferior, representada anatomicamente com o crânio menor e o quadril maior do que os do esqueleto masculino, refletindo a crença vigente de que possuiria menor capacidade intelectual e estaria naturalmente destinada somente à maternidade (Fernandes, 2009; Jackson, 2021). 

Historicamente, participantes do sexo feminino foram excluídos de estudos clínicos devido aos efeitos do ciclo menstrual aumentarem a complexidade e os custos da pesquisa e ao receio de que o tratamento investigado afetasse a fertilidade feminina ou a gravidez, além da suposta maior incidência de algumas doenças entre homens (Couto, 2021). Essa falta de inclusão perpetuada ao longo dos séculos deu origem a uma medicina com compreensão limitada da saúde feminina, o que pode levar a diagnósticos incorretos, tratamentos inadequados e negligência de sintomas ou efeitos colaterais. Porém, para além dos médicos, cientistas e pesquisadores também são majoritariamente homens, bem como a maioria das células, animais e corpos estudados na ciência são pertencentes ao sexo masculino (Jackson, 2021).

A tendência da falta de conhecimento médico e científico se torna ainda mais evidente em muitas condições predominantemente femininas. Por exemplo, mulheres com endometriose frequentemente recebem informações de que o adiamento da maternidade é a causa da doença, ou que a gravidez pode ser a cura, refletindo a antiga crença de que  a maternidade, vista como a verdadeira essência das mulheres, constituía o remédio para seus distúrbios (Fernandes, 2009). Mulheres com câncer de mama também já foram orientadas dessa mesma maneira até que avanços na pesquisa, que só ocorreram devido às campanhas realizadas por mulheres, demonstraram o contrário (Jackson, 2021).

Desde a década de 1990, houve um esforço para incluir mais mulheres nos ensaios clínicos, originado em campanhas de mulheres cientistas nos Estados Unidos que pediam por melhores pesquisas em saúde da mulher. Em resposta, em 1993 as agências governamentais norte-americanas FDA (Food and Drug Administration) e NIH (National Institutes of Health) passaram a exigir análises específicas de segurança e eficácia por sexo para todas as novas aplicações de medicamentos. Além disso, mulheres em idade fértil passaram a ser permitidas nas fases iniciais dos ensaios clínicos, removendo uma restrição significativa que anteriormente limitava a participação feminina (Jackson, 2021; Berg, 1997).  

Essa não foi a primeira vez na história recente que as mulheres influenciaram as prioridades da produção científica com relação à saúde feminina. Algumas décadas antes, a criação da primeira pílula anticoncepcional, nos anos 1960, foi resultado de uma pesquisa financiada por uma mulher feminista, a norte-americana Margaret Sanger, fundadora da Planned Parenthood Federation of America (PPFA). Na década de 40, o biólogo Gregory Pincus estudava o efeito anti-ovulatório de hormônios sexuais em animais. O impulso para converter as descobertas obtidas com seus experimentos em contracepção hormonal para humanos foi dado por Sanger, em 1951. A partir disso, Pincus passou a colaborar com John Rock, ginecologista e especialista em fertilidade, o que resultou no desenvolvimento da primeira pílula anticoncepcional, aprovada em 1960, que continha versões primitivas de progesterona e estrogênio sintéticos (Dhont, 2010). 

O estradiol, principal estrogênio humano natural, possui baixa absorção quando administrado oralmente, além de ser rapidamente inativado pelo fígado. O químico Hans Herloff Inhoffen e o endocrinologista Walter Hohlweg sintetizaram na Schering, farmacêutica alemã posteriormente adquirida pela Bayer, um estradiol com substituição na posição C17 por um grupo etinil. O resultado foi o etinilestradiol, muito mais resistente à degradação, e que, ao longo das décadas seguintes, foi consolidado como o estrogênio padrão na maioria das pílulas contraceptivas. Houve adaptações necessárias à formulação das pílulas, realizadas com o objetivo de minimizar seus efeitos colaterais, entre os quais está o risco de tromboembolismo. Contudo, a evolução das formulações, utilizando doses menores de hormônios e novos esquemas de administração, ajudou a reduzir os riscos e aumentar a adesão ao uso contínuo (Dhont, 2010).

O advento tecnológico da pílula hormonal foi uma inovação que permitiu às mulheres maior controle sobre a reprodução e foi interpretada como uma importante conquista no processo de busca pela igualdade de gênero, pois as libertaria da maternidade compulsória (Cabral, 2017). Além disso, a utilização de etinilestradiol em conjunto com outros hormônios em um regime contínuo tem se mostrado eficaz na redução das dores associadas à endometriose, melhorando significativamente a qualidade de vida das pacientes (Caruso et al., 2016; Maiorana et al., 2023). Também é observado um efeito protetor contra o câncer de endométrio e de ovário (Stanczyk et al., 2013).

Nas últimas décadas, o Brasil apresentou alta queda em sua fecundidade, porém essa queda aconteceu entre as mulheres com maior escolaridade e habitantes das zonas urbanas. Mais da metade das gestações que ocorrem no Brasil ainda não são planejadas, sendo que as mulheres que planejam suas gestações são, em sua maioria, brancas, com maior nível de instrução escolar, com mais de 35 anos de idade e se encontram em um relacionamento estável. Além disso, o aborto se configura como a quinta causa de morte materna no país (Trindade et. al, 2021).

Segundo o Ministério da Saúde (s.d.), são disponibilizados, na atenção primária à saúde e nos serviços secundários, os seguintes métodos contraceptivos: DIU de cobre, anticoncepcional oral combinado, anticoncepcional injetável combinado (aplicado mensalmente), anticoncepcional injetável de progestágeno (aplicado trimestralmente), pílula de progestágeno isolado, implante subdérmico e preservativos. Porém, as limitações enfrentadas pelas mulheres no Brasil, sobretudo as mais pobres e menos escolarizadas, para obter acesso a esses e outros métodos persistem. Apresentam-se dificuldades de diversas ordens para o acesso, sendo que há uma parcela das mulheres brasileiras que não faz uso de nenhum método contraceptivo porque não sabe aonde ir, a quem procurar para ter informações, ou não sabe como utilizá-los (Cabral, 2017; Trindade et. al, 2021).

Os métodos contraceptivos de longa duração, como o DIU, são considerados os mais eficazes e econômicos, além de não dependerem da usuária para garantir sua efetividade. Porém, mesmo com tantos benefícios, o DIU é utilizado por apenas 2 a cada 100 brasileiras, sendo mais adotado pelas mulheres com maior renda e que possuem plano de saúde (Trindade et. al, 2021). Isso pode se dever aos altos valores dos DIUs hormonais, que podem chegar a cerca de R\$1600 no mercado (ANVISA, 2024). Porém, mesmo o SUS disponibilizando o DIU de cobre para as usuárias, possivelmente há empecilhos para a sua utilização. Nesse contexto, algumas hipóteses são: mitos sobre sua eficácia e funcionamento, critérios falsos de contraindicação, necessidade de profissionais especializados para sua inserção, questões religiosas, dificuldade no acesso a exames e consultas para acompanhamento e a falta de informações sobre seus benefícios e sua ação (Trindade et. al, 2021).

De acordo com o estudo de Trindade et. al (2021), a partir de dados da Pesquisa Nacional de Saúde, 34,2\% das mulheres entrevistadas que utilizavam algum método contraceptivo relataram o uso de contraceptivo hormonal oral, popularmente conhecido como pílula anticoncepcional. Esse é o método mais utilizado pelas brasileiras, mesmo que tenha sido constatado que cerca de 20\% das mulheres que o utilizam não deveriam fazê-lo, por possuírem alguma contraindicação para seu uso. Um dos possíveis motivos para isso é que a escolha do método é frequentemente mediada pela experiência de amigas ou parentes, e as farmácias privadas permanecem como principal fonte de obtenção. A pílula, assim como os outros métodos contraceptivos, possui efeitos colaterais, que são particulares a cada mulher. Esses efeitos acarretam custos biológicos e físicos nas trajetórias contraceptivas femininas, e constituem justificativas comuns para interrupções, descontinuidades e mudanças de métodos, expondo a mulher à chance de gestação não planejada (Cabral, 2017).

Os métodos contraceptivos utilizados pelas brasileiras estão relacionados diretamente a variáveis socioeconômicas e é um fato que existem desigualdades de acesso à contracepção no país. Sendo assim, é necessário um maior investimento em políticas públicas que ampliem o acesso e o conhecimento no campo da saúde sexual e reprodutiva para as mulheres, principalmente em situação de vulnerabilidade social (Trindade et. al, 2021). Dessa forma, é muito importante que haja a compreensão de como ocorre a liberação dos hormônios no corpo e da importância de se realizar a reposição corretamente. 

\section{Objetivo}\label{sec:goal}

O objetivo geral deste trabalho é desenvolver e validar modelos matemáticos para a liberação do etinilestradiol em diferentes matrizes, sendo elas comprimidos de liberação prolongada (pílula anticoncepcional), anel vaginal e adesivo transdérmico. Através da modelagem matemática e computacional, busca-se utilizar princípios de difusão e transferência de massa para descrever a liberação do etinilestradiol em cada matriz, e, então, comparar os resultados obtidos a fim de classificar as matrizes de acordo com sua eficácia e segurança.

Tem-se, como objetivos específicos, a modelagem matemática da difusão mássica do etinilestradiol nas matrizes estudadas, o desenvolvimento de modelos computacionais para simular a difusão e a liberação nessas matrizes, a validação dos modelos desenvolvidos com dados experimentais existentes e a análise dos resultados obtidos.

Ao atingir esses objetivos, espera-se proporcionar uma reflexão sobre a contribuição da engenharia química na eficácia e segurança dos tratamentos oferecidos às mulheres, e na busca por preencher a lacuna histórica na pesquisa sobre temas relacionados à saúde feminina.