\chapter{Conclusão}\label{chp:conclusao}

Neste trabalho, foram desenvolvidos e analisados modelos computacionais para a liberação controlada de etinilestradiol em três diferentes matrizes: adesivo transdérmico, anel vaginal e contraceptivo oral combinado (COC). Utilizando a combinação de modelagem matemática, métodos numéricos e modelagem computacional, foi possível obter perfis de liberação que refletem as características específicas de cada sistema.

Para o adesivo transdérmico, foi observada uma liberação sustentada e gradual de etinilestradiol ao longo de sete dias, o que é ideal para manter níveis terapêuticos constantes no corpo. Este comportamento é consistente com a necessidade de fornecer uma dose constante ao longo do tempo, minimizando flutuações. Para o anel vaginal, foi identificada uma liberação inicial rápida seguida por uma estabilização em uma concentração mais baixa, um perfil muito similar com o descrito na literatura. Já para o contraceptivo oral combinado (COC), a liberação ao longo de 24 horas mostrou uma diminuição gradual e altas oscilações, refletindo a necessidade de administração diária para manter níveis terapêuticos. Este perfil é consistente com a prática de uso de contraceptivos orais.

Ao comparar os resultados das simulações com dados experimentais disponíveis na literatura, foi observado que as concentrações liberadas obtidas foram muito mais altas do que as relatadas na literatura para concentrações plasmáticas após administração. Essa discrepância pode ser atribuída ao fato de os modelos não levarem em consideração o metabolismo, a distribuição, a eliminação e a absorção do fármaco no corpo humano, nem a resistência da pele no modelo do adesivo transdérmico. Além disso, outra limitação é a questão dos coeficientes de difusão e parâmetros farmacocinéticos utilizados, que podem diferir dos valores reais, afetando a precisão dos resultados.

Em conclusão, este trabalho forneceu uma base inicial para a compreensão dos mecanismos de liberação e difusão de etinilestradiol em diferentes matrizes, as vantagens e limitações de cada um. De acordo com os resultados obtidos e com as informações da literatura, pode-se sugerir que haja o incentivo por parte das políticas públicas à consideração do adesivo transdérmico e do anel vaginal como métodos contraceptivos alternativos à pílula anticoncepcional. Esses métodos, apesar de mais caros, liberam o hormônio em níveis mais baixos e mais estáveis, sendo boas opções para mulheres que não se adaptarem à dosagem diária ou aos efeitos colaterais da pílula. 

A integração de processos metabólicos e ajustes nos parâmetros do presente modelo são importantes para aprimorar a precisão e a aplicabilidade dos resultados, contribuindo para o desenvolvimento de métodos contraceptivos mais eficazes e seguros.