\chapter{Código desenvolvido}

\section{Modelagem para a difusão no adesivo transdérmico}
\begin{verbatim}
import numpy as np
import matplotlib.pyplot as plt
import pandas as pd
from scipy.sparse import diags
from scipy.sparse.linalg import spsolve

# Dados
L1 = 0.004  # Espessura da matriz (cm)
D1 = 0.0000005184 / 900000  # Coeficiente de difusão (cm^2/h)
C10 = 9375  # Concentração inicial de EE (ug/cm^3)
Km = 1  # Coeficiente de partição matriz-pele (s.u.)
V1 = 0.08  # Volume total do adesivo (cm³)

# Número de pontos
Nx1 = 100  # Número de pontos espaciais

# Variáveis
T1 = 168  # Tempo total coberto na simulação (h)
dx1 = L1 / Nx1  # Passo espacial

# Passo temporal
dt = 0.1

# Concentrações iniciais
C1 = np.zeros((Nx1, int(T1/dt) + 1))
C1[:, 0] = C10

# Quantidade total de fármaco disponível no adesivo
total_farmaco_disponivel_adesivo = C10 * V1

# Coeficientes para o método de Crank-Nicolson
alpha1 = D1 * dt / (2 * dx1**2)

# Matrizes tridiagonais
A1 = diags([-alpha1, 1 + 2*alpha1, -alpha1], [-1, 0, 1], \
            shape=(Nx1-2, Nx1-2))
A1 = A1.tocsc()  # Conversão para o formato CSC
B1 = diags([alpha1, 1 - 2*alpha1, alpha1], [-1, 0, 1], \
            shape=(Nx1-2, Nx1-2))

# Método de Crank-Nicolson
total_released = 0
for n in range(0, int(T1/dt)):

    # Resolver para C1
    b1 = B1 @ C1[1:-1, n]
    C1[1:-1, n + 1] = spsolve(A1, b1)
    
    # Condição de contorno de fluxo zero para x=0
    C1[0, n + 1] = C1[1, n + 1]

    # Condição de contorno utilizando o coeficiente 
    # de partição para x=L1
    C1[-1, n + 1] = Km * C1[-2, n + 1]

    # Calcular a quantidade total de fármaco liberado a cada dia
    if (n + 1) % int(24/dt) == 0:
        daily_released = np.sum(C1[:, n + 1] - \
                        C1[:, n + 1 - int(24/dt)]) * dx1
        if daily_released > 20:
            # Parar a liberação de fármaco para o restante do dia
            C1[:, n + 1] = C1[:, n + 1 - int(24/dt)]
        total_released += daily_released

        # Verificar se a quantidade total de fármaco liberado 
        # ultrapassa a quantidade disponível
        if total_released > total_farmaco_disponivel_adesivo:
            print("A quantidade total de fármaco liberado" 
                "ultrapassa a quantidade disponível no adesivo.")
            C1[:, n + 1:] = 0
            break

# Converter tempo para dias
time_days = np.linspace(0, 7, int(T1/dt) + 1)

# Conversão de cm para µm
space_um_vehicle = np.linspace(0, L1 * 1e4, Nx1)

# Selecionar pontos temporais específicos para exibir na tabela
time_points = [int(24/dt * day) for day in range(0, 8)]
time_labels = [f't={day:.2f} dias' for day in range(0, 8)]

# Criar DataFrames para armazenar os resultados
results_vehicle = pd.DataFrame(C1[:, time_points], \
                                columns=time_labels)
results_vehicle.index = np.linspace(0, L1, Nx1)

# Exibir as tabelas
from IPython.display import display
print(
    "Perfis de Concentração na Camada Adesiva em Pontos" 
    "Temporais Selecionados:")
display(results_vehicle)

# Plotar o perfil de difusão na matriz
from mpl_toolkits.axes_grid1.inset_locator import inset_axes

# Plot principal
fig, ax = plt.subplots(figsize=(14, 8))
for i, n in enumerate(time_points):
    ax.plot(space_um_vehicle, C1[:, n], \
            label=f'Dia {int(time_days[n])}')

# Configurações do plot principal
ax.set_xlabel('Posição (µm)')
ax.set_ylabel('Concentração (µg/cm³)')
ax.legend()
ax.set_title('Difusão de EE na Camada Adesiva')

# Plot com zoom
axins = inset_axes(ax, width="40%", height="40%", \
                    loc="upper right")
for i, n in enumerate(time_points):
    axins.plot(space_um_vehicle, C1[:, n])

# Definindo os limites da região ampliada
x1, x2 = 0, 1
y1, y2 = 8800, 9400
axins.set_xlim(x1, x2)
axins.set_ylim(y1, y2)

plt.show()
\end{verbatim}

\section{Modelagem para a difusão no anel vaginal}
\begin{verbatim}
import numpy as np
import matplotlib.pyplot as plt
import pandas as pd
from scipy.sparse import diags
from scipy.sparse.linalg import spsolve

# Dados
L2 = 0.4  # Espessura da matriz (cm)
D2 = 0.00000792/10 # Coeficiente de difusão na matriz (cm^2/h)
C20 = 25.33 # Concentração inicial de EE na matriz (ug/cm^3)
V2 = 106.58  # Volume total do adesivo (cm³)

# Número de pontos
Nx2 = 100  # Número de pontos espaciais na matriz

# Variáveis
T2 = 504  # Tempo total coberto na simulação (h)
dx2 = L2 / Nx2  # Passo espacial para a matriz

# Passo temporal
dt2 = 0.1

# Concentrações iniciais
C2 = np.zeros((Nx2, int(T2/dt2) + 1))
C2[:, 0] = C20

# Quantidade total de fármaco disponível no anel
total_farmaco_disponivel_anel = C20 * V2

# Coeficientes para o método de Crank-Nicolson
alpha = D2 * dt2 / (2 * dx2**2)

# Matrizes tridiagonais
A2 = diags([-alpha, 1 + 2*alpha, -alpha], [-1, 0, 1], \
        shape=(Nx2-2, Nx2-2))
A2 = A2.tocsc()
B2 = diags([alpha, 1 - 2*alpha, alpha], [-1, 0, 1], \
        shape=(Nx2-2, Nx2-2))

# Método de Crank-Nicolson
for n in range(0, int(T2/dt2)):
    if n % 100 == 0:
        print(f"Iteração {n} de {int(T2/dt2)}")
    # Resolver para C
    b2 = B2 @ C2[1:-1, n]
    C2[1:-1, n + 1] = spsolve(A2, b2)

    # Condição de contorno de fluxo zero para x=0
    C2[0, n + 1] = C2[1, n + 1]

    # Condição de contorno para x=L
    C2[-1, n + 1] = 0

    # Calcular a quantidade total de fármaco liberado a cada dia
    if (n + 1) % int(24/dt) == 0:
        daily_released = np.sum(C2[:, n + 1] - \
                        C2[:, n + 1 - int(24/dt)]) * dx2
        if daily_released > 15:
            # Parar a liberação de fármaco para o restante do dia
            C2[:, n + 1] = C2[:, n + 1 - int(24/dt)]
        total_released += daily_released

        # Verificar se a quantidade total de fármaco liberado 
        # ultrapassa a quantidade disponível
        if total_released > total_farmaco_disponivel_anel:
            print("A quantidade total de fármaco liberado" 
                "ultrapassa a quantidade disponível no anel.")
            C2[:, n + 1:] = 0
            break

# Converter tempo para semanas
num_weeks = T2 // (24 * 7)
time_weeks = np.linspace(0, num_weeks, int(T2/dt2) + 1)

# Selecionar pontos temporais específicos para exibir na tabela
time_points = [int((7 * 24) / dt2) * \
            week for week in range(num_weeks + 1)]
time_labels = [f'Semana {week}' for week in range(num_weeks + 1)]

# Criar DataFrames para armazenar os resultados
results_vehicle = pd.DataFrame(C2[:, time_points], \
                        columns=time_labels)
results_vehicle.index = np.linspace(0, L2, Nx2)

# Exibir as tabelas
from IPython.display import display
print("Perfis de Concentração na Camada Polimérica" 
    "em Pontos Temporais Selecionados:")
display(results_vehicle)

# Plotar o perfil de difusão
plt.figure(figsize=(12, 6))
for tp in time_points:
    plt.plot(np.linspace(0, L2, Nx2), C2[:, tp], \
            label=f'{time_labels[time_points.index(tp)]}')
plt.xlabel('Posição (cm)')
plt.ylabel('Concentração (ug/cm^3)')
plt.legend()
plt.title('Difusão de EE na Camada Polimérica')
plt.show()
\end{verbatim}

\section{Modelagem para a liberação a partir do COC}
\begin{verbatim}
import numpy as np
import matplotlib.pyplot as plt

# Dados
Cs = 18.7  # Solubilidade máxima do EE no fluido gastrointestinal (ug/cm^3)
kd = 0.02  # Constante de dissolução (1/h)
T3 = 24  # Tempo total coberto na simulação (h)
dt = 0.1  # Passo temporal (h)
C_max = 15  # Concentração máxima permitida (ug/cm^3)

# Função que representa a EDO
def dCdt(Cd, t):
    return kd * (Cs - Cd)

# Inicialização
t_values = np.arange(0, T3 + dt, dt)
C3 = np.zeros(len(t_values))
C3[0] = 0  # Concentração inicial de EE no fluido gastrointestinal

# Método de Runge-Kutta de quarta ordem (RK4) com limite de concentração
for i in range(1, len(t_values)):
    t = t_values[i-1]
    Cd = C3[i-1]

    k1 = dt * dCdt(Cd, t)
    k2 = dt * dCdt(Cd + 0.5 * k1, t + 0.5 * dt)
    k3 = dt * dCdt(Cd + 0.5 * k2, t + 0.5 * dt)
    k4 = dt * dCdt(Cd + k3, t + dt)

    # Calcular a nova concentração e impor o limite
    C3[i] = Cd + (k1 + 2*k2 + 2*k3 + k4) / 6
    if C3[i] > C_max:
        C3[i] = C_max

# Plotar os resultados
plt.figure(figsize=(12, 6))
plt.plot(t_values, C3, label='Concentração liberada')
plt.xlabel('Tempo (h)')
plt.ylabel('Concentração (ug/cm^3)')
plt.legend()
plt.title('Liberação de EE a partir da Pílula')
plt.show()
\end{verbatim}

\section{Concentração liberada em função do tempo para cada matriz}

\begin{verbatim}
import numpy as np
import matplotlib.pyplot as plt

# Cálulo das taxas para cada matriz

def calculate_release_rate(C, dx=None):
    release_rate = np.zeros(C.shape[1]) if C.ndim == 2 else \
                    np.zeros(len(C))
    for n in range(1, C.shape[1] if C.ndim == 2 else len(C)):
        if C.ndim == 2: # Adesivo e anel
            # Usar a média da concentração ao longo da matriz 
            # para cada ponto temporal
            avg_concentration = np.mean(C[:, n] - C[:, n - 1])
            rate = avg_concentration
        else: # COC
            dC = C[n] - C[n - 1]
            rate = dC
        release_rate[n] = rate
    return np.abs(release_rate)

# Calculando a taxa
release_rate_patch = calculate_release_rate(C1, dx1)
release_rate_ring = calculate_release_rate(C2, dx2)
release_rate_pill = calculate_release_rate(C3)

# Criando subplots para plotar cada matriz separadamente
fig, axs = plt.subplots(1, 3, figsize=(18, 6))

# Plotando resultados para o adesivo
axs[0].plot(np.linspace(0, T1, len(release_rate_patch)), \
            release_rate_patch, label='Taxa de Liberação - Adesivo')
axs[0].set_xlabel('Tempo (h)')
axs[0].set_ylabel('Taxa de Liberação (ug/mL)')
axs[0].set_title('Adesivo Transdérmico')
axs[0].legend()

# Plotando resultados para o anel vaginal
axs[1].plot(np.linspace(0, T2, len(release_rate_ring)), \
            release_rate_ring, label='Taxa de Liberação - Anel')
axs[1].set_xlabel('Tempo (h)')
axs[1].set_ylabel('Taxa de Liberação (ug/mL)')
axs[1].set_title('Anel Vaginal')
axs[1].set_xlim(0, 1)  # Recorte de tempo para a primeira hora
axs[1].legend()

# Plotando resultados para o COC
axs[2].plot(np.linspace(0, T3, len(release_rate_pill)), \
            release_rate_pill, label='Taxa de Liberação - COC')
axs[2].set_xlabel('Tempo (h)')
axs[2].set_ylabel('Taxa de Liberação (ug/mL)')
axs[2].set_title('Comprimido Oral Combinado')
axs[2].legend()

# Adicionar título unificado
fig.suptitle('Taxa de Liberação de Etinilestradiol', fontsize=16)

# Ajustar layout
plt.tight_layout()
plt.show()

# Plotando as matrizes juntas e simulando reposição

# Parâmetros
dt = 0.1  # Passo temporal (h)
T_total = 21 * 24  # Tempo total de 21 dias em horas
num_days = 21  # Número total de dias

# Função para empilhar as curvas simulando a reposição
def stack_curves(data, repeat_interval, total_time, dt):
    num_repeats = int(total_time / repeat_interval)
    stacked_data = np.zeros(int(total_time / dt))
    for i in range(num_repeats):
        start_idx = int(i * repeat_interval / dt)
        end_idx = start_idx + len(data)
        stacked_data[start_idx:end_idx] = data
    return stacked_data

# Empilhar as curvas
release_rate_pill_stacked = \
    stack_curves(release_rate_pill[:int(24/dt)],24, T_total, dt)
release_rate_patch_stacked = \
    stack_curves(release_rate_patch[:int(168/dt)], 168, T_total, dt)
release_rate_ring_stacked = \
    stack_curves(release_rate_ring[:int(504/dt)], 504, T_total, dt)

# Convertendo as taxas de liberação de µg/h para pg/h
release_rate_pill_stacked *= 1e6
release_rate_patch_stacked *= 1e6
release_rate_ring_stacked *= 1e6

# Redução do número de pontos no eixo x 
# para evitar erro de tamanho da imagem
num_points = 1000
x_values = np.linspace(0, num_days, num_points)
release_rate_pill_reduced = np.interp(x_values, \
    np.linspace(0, num_days, len(release_rate_pill_stacked)), \
    release_rate_pill_stacked)
release_rate_patch_reduced = np.interp(x_values, \
    np.linspace(0, num_days, len(release_rate_patch_stacked)), \
    release_rate_patch_stacked)
release_rate_ring_reduced = np.interp(x_values, np.linspace(0, num_days, \
len(release_rate_ring_stacked)), release_rate_ring_stacked)

# Definindo cores personalizadas
color_ring = '#1f77b4'
color_patch = '#ff7f0e'
color_pill = '#2ca02c'

# Plotar os resultados
plt.figure(figsize=(12, 6))
plt.plot(x_values, release_rate_pill_reduced, label='COC', color=color_pill)
plt.plot(x_values, release_rate_patch_reduced, label='Adesivo', \
            color=color_patch)
plt.plot(x_values, release_rate_ring_reduced, label='Anel Vaginal', \
            color=color_ring)
plt.xlabel('Tempo (dias)')
plt.ylabel('Concentração liberada (pg/h)')
plt.title('Liberação de etinilestradiol ao longo do tempo')
plt.legend()
plt.show()
\end{verbatim}