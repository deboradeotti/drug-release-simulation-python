\begin{resumo}
A exclusão histórica das mulheres da pesquisa médica resultou em vieses na compreensão das questões de saúde feminina. Este trabalho visa aprimorar os conhecimentos relativos aos métodos de liberação de etinilestradiol (EE), um estrogênio sintético utilizado em diversos métodos contraceptivos. Para isso, foi realizada a modelagem matemática e computacional da liberação controlada de etinilestradiol em diferentes matrizes de liberação, especificamente o adesivo transdérmico, anel vaginal e comprimido oral combinado (COC), conhecido popularmente como pílula anticoncepcional. Cada sistema de liberação possui características próprias de difusão e liberação controlada, atendendo a necessidades específicas de administração de hormônios em contraceptivos combinados. Para o adesivo transdérmico e o anel vaginal, foi utilizado um modelo de difusão baseado em equações diferenciais parciais (EDPs) para simular o transporte de EE através das camadas da matriz. As condições de contorno pertinentes às restrições de cada sistema foram aplicadas, e as equações foram resolvidas através do método numérico de Crank-Nicolson utilizando Python. Para a pílula anticoncepcional, foi utilizada a equação de Noyes-Whitney, descrevendo a dissolução do EE no fluido gastrointestinal por meio de um modelo cinético de primeira ordem. A EDO correspondente foi resolvida utilizando o método de Runge-Kutta de quarta ordem (RK4). Foram plotados gráficos para ilustrar a difusão do EE ao longo das camadas poliméricas e sua liberação a partir de cada uma das matrizes. Os resultados indicam que o adesivo transdérmico proporciona uma liberação sustentada de EE ao longo de sete dias, enquanto o anel vaginal exibe uma liberação inicial rápida seguida de estabilização ao longo de três semanas. O COC, por sua vez, apresenta concentrações muito mais altas e flutuações significativas na liberação, necessitando de administração diária para manter níveis terapêuticos adequados. As limitações do estudo incluem a não consideração dos processos metabólicos e da resistência da pele. A análise de sensibilidade realizada para o coeficiente de difusão \( D \) revelou que variações nesse parâmetro afetam significativamente a taxa de liberação do fármaco. Valores maiores de \( D \) resultam em uma difusão mais rápida, enquanto valores menores resultam em uma difusão mais lenta, impactando diretamente o desempenho do sistema de liberação. Concluiu-se que a modelagem computacional é uma ferramenta valiosa para otimizar os mecanismos de liberação de medicamentos, melhorando a segurança e a eficácia dos métodos contraceptivos e contribuindo para melhores condições de saúde para as mulheres.
\end{resumo}